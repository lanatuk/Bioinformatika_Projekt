\documentclass[times, utf8, seminar]{fer}
\usepackage{booktabs}

\begin{document}

% TODO: Navedite naslov rada.
\title{Poboljšanje djelomično sastavljenog genoma dugim očitanjima}

% TODO: Navedite vaše ime i prezime.
\author{Lana Tuković, Ema Vlainić}

% TODO: Navedite ime i prezime mentora.
\voditelj{Krešimir Križanović}

\maketitle

\tableofcontents

\chapter{Uvod}
Postupak sastavljanja složenih genoma može dati fragmentiran rezultat zbog velikog broja ponavljajućih sekvenci koje otežavaju proces poravnanja. Naš zadatak bio je pokušati međusobno povezati dobivene fragmente - contige u cijeli genom. Algoritam koji smo pri tome koristili zasniva se na konstruiranju grafa preklapanja te pronalaženja optimalnih staza među preklapanjima. Pronađena optimalna staza će nam služiti da povežemo dva contig-a.

\chapter{Opis algoritma}
Skupovi očitanja i već sastavljenih contig-a pripremljeni su kao testni podaci. Njihova preklapanja dobivena pomoću alata Minimap2 koristimo kao ulazne točke programa.

\underline{Ulazni podaci:}
\begin{itemize}
	\item[•]{preklapanja između contig-a i očitanja u PAF formatu dobivena korištenjem alata Minimap2 nad datotekom sa skupom contig-a i datotekom sa skupom očitanja}
	\item[•]{međusobna preklapanja očitanja u PAF formatu dobivena korištenjem alata Minimap2 nad dvije iste datoteke sa skupom očitanja}
\end{itemize}

\underline{Izlazni podaci:}
\begin{itemize}
	\item[•]{poboljšani skup sastavljenih contiga u FASTA formatu}
\end{itemize}

\begin{flushleft}
Sada kada imamo sve potrebne podatke možemo konstruirati graf preklapanja. Graf preklapanja sastoji se od dvije vrste čvorova: usidreni čvorovi (\textit{anchoring nodes}) koji predstavljaju unaprijed sastavljane contig-e i čvorovi očitanja (\textit{read nodes}). Veza između dva čvorova predstavlja preklapanje tih dvaju čvorova. Kada je graf konstruiran, slijedi traženje optimalnih staza iz svakog usidrenog čvora do skupa mogućih završnih usidrenih čvorova.
\end{flushleft}


\chapter{Zaključak}
Zaključak.

\bibliography{literatura}
\bibliographystyle{fer}

\end{document}
